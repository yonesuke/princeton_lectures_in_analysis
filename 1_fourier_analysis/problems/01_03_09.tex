\begin{renshu}
摘み上げた弦の場合、フーリエ・サイン係数に対する公式を用いて
\begin{align}
A_m=\frac{2h}{m^{2}}\frac{\sin mp}{p(\pi - p)}
\end{align}
を示せ。$p$がどのような位置にあるとき、第2,4,...のハーモニクスは失われるか?
また、$p$がどのような位置にあるとき、第3,6,...のハーモニクスは失われるか?
\end{renshu}

\begin{kaitou*}
弦がはじめに次の三角形の形をしている状態を考える。
\begin{align*}
f(x)=\left\{
\begin{array}{ll}
\dfrac{xh}{p}, & 0\le x\le p,\\
\dfrac{h(\pi-x)}{\pi-p}, & p\le x\le \pi
\end{array}
\right.
\end{align*}
これに対して、$A_{m}=(2/\pi)\int_{0}^{\pi}f(x)\sin mxdx$を計算する\footnote{$x\sin ax=[-(1/a)x\cos ax+(1/a^{2})\sin ax]'$}。
\begin{align*}
A_{m}=&\frac{2}{\pi}\frac{h}{p}\int_{0}^{p}x\sin mxdx+\frac{2}{\pi}\frac{h\pi}{\pi-p}\int_{p}^{\pi}\sin mxdx-\frac{2}{\pi}\frac{h}{\pi-p}\int_{p}^{\pi}x\sin mxdx\\
=&\frac{2h}{\pi p}\left(-\frac{1}{m}p\cos mp+\frac{1}{m^2}\sin mp\right)
+\frac{2h}{\pi-p}\left(-\frac{1}{m}\cos m\pi+\frac{1}{m}\cos mp\right)\\
&-\frac{2h}{\pi(\pi-p)}\left(-\frac{1}{m}\pi\cos m\pi+\frac{1}{m}p\cos mp-\frac{1}{m^2}\sin mp\right)\\
=&\frac{2h}{m^{2}}\frac{\sin mp}{p(\pi - p)}
\end{align*}

$\sin 2p=\sin 4p=\sin6p=\cdots=0$となる$p$は$p=\pi/2$となるときである。また、$\sin 3p=\sin 6p=\sin 9p=\cdots=0$となる$p$は$p=\pi/3$もしくは$p=2\pi/3$となる。

\end{kaitou*}