\begin{renshu}
複素数$z=x+iy,\ x,y\in\mathbb{R}$に対して、
\begin{align}
    |z|=(x^{2}+y^{2})^{1/2}
\end{align}
と定義し、これを$z$の\textbf{絶対値}という。
\begin{enumerate}
\item $|z|$の幾何的な意味は何か?
\item $|z|=0$ならば$z=0$であることを示せ。
\item $\lambda\in\mathbb{R}$であれば、$|\lambda z|=|\lambda||z|$を示せ。
ただし、$|\lambda|$は実数に対する通常の絶対値を表す。
\item $z_{1},z_{2}$を複素数とするとき
\begin{align}
    |z_{1}z_{2}|=|z_{1}||z_{2}|,\quad |z_{1}+z_{2}|\leq|z_{1}|+|z_{2}|
\end{align}
を証明せよ。
\item $z\ne0$のとき$|1/z|=1/|z|$を示せ。
\end{enumerate}
\end{renshu}

\begin{kaitou*}
\begin{enumerate}
\item $|z|$は複素平面上における原点から$z$までの距離を表す。
\item $|z|=0$のとき、$x^{2}+y^{2}=0$である。これを満たす実数$x,y$は$x=y=0$であり、これより$z=0$である。
\item \begin{align*}
|\lambda z|=|\lambda x+i\lambda y|=(\lambda^{2}x^{2}+\lambda^{2}y^{2})^{1/2}
=|\lambda|(x^{2}+y^{2})^{1/2}=|\lambda||z|
\end{align*}
\item $z_{i}=x_{i}+iy_{i},\ x_{i},y_{i}\in\mathbb{R}\ (i=1,2)$とおく。
\begin{align*}
|z_{1}z_{2}|=(x_{1}^{2}x_{2}^{2}+y_{1}^{2}y_{2}^{2}+x_{1}^{2}y_{2}^{2}+x_{2}^{2}y_{1}^{2})^{1/2}=|z_{1}||z_{2}|
\end{align*}
がまず示される。また、
\begin{align*}
(|z_{1}|+|z_{2}|)^{2}-|z_{1}+z_{2}|^{2}=2[(x_{1}^{2}+y_{1}^{2})^{1/2}(x_{2}^{2}+y_{2}^{2})^{1/2}-(x_{1}x_{2}+y_{1}y_{2})]\geq0
\end{align*}
であるから、$|z_{1}+z_{2}|\leq|z_{1}|+|z_{2}|$である。
ここでコーシーシュワルツの不等式
\footnote{コーシーシュワルツの不等式は$\mathbf{a}_{1},\mathbf{a}_{2}\in\mathbb{R}^{n}$に対して$|\mathbf{a}_{1}||\mathbf{a}_{2}|\geq|\mathbf{a}_{1}\cdot\mathbf{a}_{2}|$である。}
を用いた。
\item $z\ne0$のとき
\begin{align*}
|z||1/z|=|z\cdot 1/z|=1
\end{align*}
である。$|z|\ne0$であるから$|1/z|=1/|z|$である。
\end{enumerate}
\end{kaitou*}