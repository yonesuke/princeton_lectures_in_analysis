\begin{renshu}
整数 $n$ に対して $f(x) = e^{inx}$ が周期 $2\pi$ の周期関数であることと
\begin{align}
\frac{1}{2\pi} \int_{-\pi}^{\pi} e^{inx} dx = \left\{
\begin{array}{ll}
1, & n=0 \\
0, & n \ne 0
\end{array}
\right.
\end{align}
を証明せよ。
このことを用いて、整数 $n, m \ge 1$ に対して
\begin{align}
\frac{1}{\pi}\int_{-\pi}^{\pi} \cos nx \cos mx dx = \left\{
\begin{array}{ll}
0, & n \ne m \\
1, & n = m
\end{array}
\right.
\end{align}
を示せ。同様にして
\begin{align}
\frac{1}{\pi} \int_{-\pi}^{\pi} \sin nx \sin mx dx = \left\{
\begin{array}{ll}
0, & n \ne m \\
1, & n = m
\end{array}
\right.
\end{align}
を示せ。最後に任意の $n, m$ に対して
\begin{align}
\int_{-\pi}^{\pi} \sin nx \cos mx dx = 0
\end{align}
を示せ。
\end{renshu}

\begin{kaitou*}
$f(x+2\pi)=e^{in(x+2\pi)}=e^{inx}e^{2\pi ni}=e^{inx}=f(x)$より$f$は周期$2\pi$の周期関数である。
$n=0$のとき$\int_{-\pi}^{\pi} e^{inx} dx=\int_{-\pi}^{\pi}dx=2\pi$であり、$n\neq 0$のとき$\int_{-\pi}^{\pi} e^{inx} dx=\int_{-\pi}^{\pi} \cos nx dx+i\int_{-\pi}^{\pi} \sin nx dx=(1/n)[\sin nx - i \cos nx]_{-\pi}^{\pi}=0$である。よって、$1/(2\pi)\int_{-\pi}^{\pi}e^{inx} dx=\delta_{n, 0}$が示された。

$\cos nx \cos mx=1/2\mathrm{Re}(e^{i(n+m)x}+e^{i(n-m)x})$より$\int_{-\pi}^{\pi}\cos nx \cos mx dx=1/2\mathrm{Re}(2\pi\delta_{n+m,0}+2\pi\delta_{n-m,0})=\pi\delta_{n,m}$が示される。

$\sin nx \sin mx=1/2\mathrm{Re}(e^{i(n-m)x}-e^{i(n+m)x})$より$\int_{-\pi}^{\pi}\sin nx \sin mx dx=1/2\mathrm{Re}(2\pi\delta_{n-m,0}-2\pi\delta_{n+m,0})=\pi\delta_{n,m}$が示される。

$\sin nx \cos mx=1/2\mathrm{Im}(e^{i(n+m)x}+e^{i(n-m)x})$より$\int_{-\pi}^{\pi}\sin nx \cos mx dx=1/2\mathrm{Im}(2\pi\delta_{n+m,0}+2\pi\delta_{n-m,0})=0$が示される。

\end{kaitou*}