\begin{renshu}
  $z=x+iy,\ x,y\in\mathbb{R}$が複素数のとき、$z$の\textbf{複素共役}を
\begin{align}
\bar{z}=x-iy
\end{align}
により定義する。
\begin{enumerate}
\item $\bar{z}$の幾何学的な意味は何か?
\item $|z|^{2}=z\bar{z}$を示せ。
\item $z$が単位円周上にあるとき、$1/z=\bar{z}$を証明せよ。
\end{enumerate}
\end{renshu}

\begin{kaitou*}
\begin{enumerate}
\item $\bar{z}$は実軸周りに$z$を反転させたものに対応する。
\item 
\begin{align*}
|z|=x^{2}+y^{2}=(x+iy)(x-iy)=z\bar{z}
\end{align*}
\item $z$が単位円周上にあるとき、$|z|=1$である。
上式に代入して、$\bar{z}=1/z$である。
\end{enumerate}
\end{kaitou*}