\begin{renshu}
  複素数列$\{w_{n}\}_{n=1}^{\infty}$が収束するとは、ある$w\in\mathbb{C}$が存在し、
\begin{align}
    \lim_{n\to\infty}|w_{n}-w|=0
\end{align}
をみたすことであり、$w$をこの数列の極限という。
\begin{enumerate}
\item 複素数列が収束するとき、その極限は一意的に定まることを示せ。

複素数列$\{w_{n}\}_{n=1}^{\infty}$が\textbf{コーシー列}であるとは、
任意の$\varepsilon>0$に対して、ある正の整数$N$で
\begin{align}
n,m>N\Rightarrow|w_{n}-w_{m}|<\varepsilon
\end{align}
を満たすようなものが存在することである。
\item 複素数列が収束するのは、それがコーシー列のとき、かつそのときに限ることを証明せよ。

複素数の級数$\sum_{n=1}^{\infty}z_{n}$が収束するとは、その部分和
\begin{align}
S_{N}=\sum_{n=1}^{N}z_{n}
\end{align}
が収束することである。$\{a_{n}\}_{n=1}^{\infty}$を非負の実数列で$\sum_{n=1}^{\infty}a_{n}$が収束するものとする。
\item $\{z_{n}\}_{n=1}^{\infty}$が複素数列で、すべての$n$に対して$|z_{n}|\leq a_{n}$をみたしているとする。
このとき$\sum_{n=1}^{\infty}z_{n}$が収束することを示せ。
\end{enumerate}
\end{renshu}

\begin{kaitou*}
\begin{enumerate}
\item $\{w_{n}\}_{n=1}^{\infty}$が収束し、その極限が$w_{1},w_{2}$とする。
定義より$k=1,2$それぞれについて、
任意の$\varepsilon>0$に対してある$N_{k}\in\mathbb{N}$が存在し、$n>N_{k}$ならば
$|w_{n}-w_{k}|<\varepsilon$となる。
これより、任意の$\varepsilon>0$に対して$N=\max\{N_{1},N_{2}\}$とおくと、
$n>N$ならば$|w_{1}-w_{2}|<2\varepsilon$となる。
よって$|w_{1}-w_{2}|=0$であり、$w_{1}=w_{2}$が示された。
\item はじめに複素数列$\{w_{n}\}_{n=1}^{\infty}$が収束列のときにCauchy列であることを示す。
$\{w_{n}\}_{n=1}^{\infty}$の極限を$w$とおくと、定義から
任意の$\varepsilon>0$に対してある$N\in\mathbb{N}$が存在して、
$n>N$ならば$|w_{n}-w|<\varepsilon$となる。
このとき、$m>N$なる$m$についても$|w_{m}-w|<\varepsilon$となる。
よって、$n,m>N$ならば$|w_{n}-w_{m}|<2\varepsilon$となるのでCauchy列となる。

次に複素数列$\{w_{n}\}_{n=1}^{\infty}$がCauchy列のときに収束列であることを示す。
はじめに$\{w_{n}\}_{n=1}^{\infty}$が有界であることを示す。
任意の$\varepsilon>0$に対して、ある正の整数$N$が存在し、$n\ge N$であれば$|w_n-w_N|<\varepsilon$が成り立つ。これより、$|w_n|<\varepsilon + |w_N|$である。
よって、$M=\max\left\{|w_1|,\dots,|w_{N-1}|,\varepsilon + |w_N|\right\}$とおくと、任意の正の整数$n$で$|w_n|<M$がわかる。これは$\{w_{n}\}_{n=1}^{\infty}$が有界であることに他ならない。
ボルツァノ・ワイエルシュトラスの定理より有界な列には収束する部分列$\{w_{n_{k}}\}_{k=1}^{\infty}$が存在する。$\lim_{k\to\infty}w_{n_{k}}=w$とおくと、任意の$\varepsilon>0$に対してある正の整数$K$が存在して、$k\ge K$ならば$|w_{n_{k}}-w|<\varepsilon$とかける。これとCauchy列の条件をそろえると、$n\ge\max\{N,n_{K}\}$であれば$|w_{n}-w|\le|w_{n}-w_{n_{K}}|+|w_{n_{K}}-w|<2\varepsilon$となる。これにより$\{w_{n}\}_{n=1}^{\infty}$が収束列であることが示された。

\item $\{\sum_{k=1}^{n}a_k\}_{k}$は収束するのでCauchy列である。よって、任意の$\varepsilon>0$に対して、ある正の整数$N$が存在して、$n\ge m>N$ならば$\sum_{k=m+1}^{n}a_k<\varepsilon$となる。このとき、$|S_{n}-S_{m}|=|\sum_{k=m+1}^{n}z_{k}|<\sum_{k=m+1}^{n}a_k<\varepsilon$が得られる。よって、$S_{n}$はCauchy列であるので、収束することが示された。
\end{enumerate}
\end{kaitou*}