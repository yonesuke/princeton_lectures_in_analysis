\begin{renshu}
  $z\in\mathbb{C}$に対して、その\textbf{複素指数}を
\begin{align}
e^{z}=\sum_{n=0}^{\infty}\frac{z^{n}}{n!}
\end{align}
により定義する。
\begin{enumerate}
\item すべての複素数に対して、この級数が収束することを証明し、上記の定義が意味をもつことを確認せよ。
さらに$\mathbb{C}$の任意の有界閉集合上で、この収束は一様収束であることを示せ。
\item $z_{1},z_{2}$が複素数であるとき、$e^{z_{1}+z_{2}}=e^{z_{1}}e^{z_{2}}$であることを示せ。
\item $z$が純虚数であるとき、すなわち$z=iy,\ y\in\mathbb{R}$であるとき、
\begin{align}
e^{iy}=\cos y+i\sin y
\end{align}
を示せ。これはオイラーの等式である。
\item 一般に$x,y\in\mathbb{R}$に対して、
\begin{align}
    e^{x+iy}=e^{x}(\cos y+i\sin y)
\end{align}
である。
\begin{align}
|e^{x+iy}|=e^{x}
\end{align}
を示せ。
\item $e^{z}=1$が成り立つのは、ある整数$k$に対して$z=2\pi ki$であるとき、かつそのときに限ることを証明せよ。
\item 複素数$z=x+iy$が次の形に書けることを示せ。
\begin{align}
z=re^{i\theta}
\end{align}
ただし$0\leq r<\infty$であり、$\theta\in\mathbb{R}$は$2\pi$の整数倍の違いを除いて一意的に定まる。
また、次の式が意味をもつとき、
\begin{align}
r=|z|,\quad\theta=\arctan(y/x)
\end{align}
であることを確認せよ。
\item 特に$i=e^{i\pi/2}$である。複素数に$i$を掛けることの幾何的な意味は何か?
また、$\theta\in\mathbb{R}$に対して$e^{i\theta}$を掛けることの幾何的な意味は何か?
\item 与えられた$\theta\in\mathbb{R}$に対して
\begin{align}
\cos\theta=\frac{e^{i\theta}+e^{-i\theta}}{2},\quad\sin\theta=\frac{e^{i\theta}-e^{-i\theta}}{2i}
\end{align}
を示せ。これらもオイラーの等式と呼ばれている。
\item 複素関数を用いて
\begin{align}
\cos(\theta+\vartheta)=\cos\theta\cos\vartheta-\sin\theta\sin\vartheta
\end{align}
などの三角関数に関する等式を示せ。それから
\begin{align}
2\sin\theta\sin\varphi=\cos(\theta-\varphi)-\cos(\theta+\varphi),\\
2\sin\theta\cos\varphi=\sin(\theta+\varphi)+\sin(\theta-\varphi)
\end{align}
を示せ。この計算はダランベールによる進行波を用いた解と定常波の重ね合わせによる解を結びつけるものである。
\end{enumerate}
\end{renshu}

\begin{kaitou*}
\begin{enumerate}
\item $S_{n}=\sum_{k=0}^{n}z^{k}/k!$として、$S_{n}$がCauchy列であることを示す。
任意の$z\in\mathbb{C}$はある$M>0$によって$|z|<M$と抑えられる。
任意の$\varepsilon>0$に対して正の整数$N$を$\varepsilon>1/2^N$かつ$N!>(2M)^{N}$\footnote{一般に$a>0$ならば$\lim_{n\to\infty}a^n/n!=0$である。}が成り立つようなものとして取る。このとき$n\geq m > N$であれば$|S_n-S_m|<\sum_{k=m+1}^{n}M^{k}/k!<1/2^{N}<\varepsilon$となり、Cauchy列であることが示された。

次に$e^z$が有界閉集合$S\subset\mathbb{C}$上で一様収束することを示す。$S$は有界であるからある$M>0$が存在して、任意の$z\in S$において$|z|<M$である。$z=M$においても$e^{M}$は収束するので、任意の$\varepsilon>0$に対して正の整数$N$が存在して、$n>N$であれば$\sum_{k=n+1}^{\infty}M^{k}/k!<\varepsilon$である。このとき、任意の$z\in S$において、$n>N$において$|\sum_{k=n+1}^{\infty}z^{k}/k!|<\sum_{k=n+1}^{\infty}|z|^{k}/k!<\sum_{k=n+1}^{\infty}M^{k}/k!<\varepsilon$がわかる。よって$S$上でこの収束は一様収束であることが示された。

\item $e^{z_1}e^{z_2}=\sum_{m=0}^{\infty}z_1^{m}/m!\sum_{n=0}^{\infty}z_2^{n}/n!=\sum_{m,n=0}^{\infty}z_1^m z_2^n/(m!n!)$において、$m+n=k$となる項をまとめると、$\sum_{k=0}^{\infty}1/k!\sum_{n=0}^{k}\binom{k}{n}z_1^{k-n}z_2^n=\sum_{k=0}^{\infty}(z_1+z_2)^k/k!=e^{z_1+z_2}$となる。よって$e^{z_1}e^{z_2}=e^{z_1+z_2}$である。これらの操作は$e^{z}$が絶対収束することから正当化される。

\item $e^{iy}=\sum_{n=0}^{\infty}i^n y^n/n!=(1-y^2/2!+y^4/4!-\cdots)+i(y/1!-y^3/3!+y^5/5!-\cdots)=\cos y+i\sin y$である。これらの操作についても$e^{z}$が絶対収束することから正当化される。

\item $e^{x+iy}=e^{x}e^{iy}=e^{x}(\cos y+i\sin y)$である。また、$|e^{x+iy}|=|e^{x}||\cos y+i\sin y|$において、$e^{x}>0$と$|\cos y+i\sin y|=\cos^{2}y+\sin^{2}y=1$より、$|e^{x+iy}|=e^{x}$である。

\item $1=e^{z}=e^{x}\cos y+i e^{x}\sin y$の虚部に注目すると$\sin y=0$、すなわち$y=\pi k, k\in\mathbb{Z}$が必要。このとき、$1=(\pm 1)^k e^x$となる。$e^{x}>0$より、$k$は偶数かつ$x=0$となる。よって、$k\in\mathbb{Z}$として$z=2\pi ki$が得られる。

\item 

\end{enumerate}
\end{kaitou*}