\begin{renshu}
$a$と$b$が実数であるとき、次のように表せることを示せ。
\begin{align}
a\cos ct + b\sin ct = A\cos(ct-\varphi)
\end{align}
ただし、$A=\sqrt{a^2+b^2}$であり、$\varphi$は
\begin{align}
\cos\varphi=\frac{a}{\sqrt{a^2+b^2}},\quad \sin\varphi=\frac{b}{\sqrt{a^2+b^2}}
\end{align}
となるように選んでいる。
\end{renshu}

\begin{kaitou*}
$A\cos(ct-\varphi)=A\cos ct \cos\varphi + A\sin ct \sin\varphi=a\cos ct + b\sin ct$
\end{kaitou*}