\begin{renshu}
$f$が$\mathbb{R}$上で2回連続微分可能で、かつ方程式
\begin{align}
f''(t)+c^2 f(t)=0
\end{align}
の解であれば、ある定数$a,b$が存在し、
\begin{align}
f(t)=a\cos ct + b\sin ct
\end{align}
となっていることを証明せよ。
これは二つの関数$g(t)=f(t)\cos ct - c^{-1}f'(t)\sin ct$と$h(t)=f(t)\sin ct + c^{-1}f'(t)\cos ct$を微分することによって示すことが出来る。
\end{renshu}

\begin{kaitou*}
$g'(t)=h'(t)=0$であるから、ある定数$a,b$が存在して$g(t)=a,h(t)=b$となる。
$g(t)\cos ct+h(t)\sin ct$を計算することで$f(t)=a\cos ct+b\sin ct$を得る。
$f'(t)=-ac\sin ct+bc\cos ct$となり、$g(t)=a,h(t)=b$を満たす。
\end{kaitou*}